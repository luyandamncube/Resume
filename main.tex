%% If you want to use \orcid or the
%% academicons icons, add "academicons"
%% to the \documentclass options. 
%% Then compile with XeLaTeX or LuaLaTeX.
% \documentclass[10pt,a4paper,academicons]{altacv}
\documentclass[10pt,a4paper]{altacv}

%% AltaCV uses the fontawesome and academicon fonts
%% and packages. 
%% See texdoc.net/pkg/fontawecome and http://texdoc.net/pkg/academicons for full list of symbols.
%% When using the "academicons" option,
%% Compile with LuaLaTeX for best results. If you
%% want to use XeLaTeX, you may need to install
%% Academicons.ttf in your operating system's font %% folder.


% Change the page layout if you need to
\geometry{left=1cm,right=9cm,marginparwidth=6.8cm,marginparsep=1.2cm,top=1cm,bottom=1cm}

% Change the font if you want to.

% If using pdflatex:
\usepackage[utf8]{inputenc}
\usepackage[T1]{fontenc}
\usepackage[default]{lato}

% If using xelatex or lualatex:
% \setmainfont{Lato}

% Change the colours if you want to
\definecolor{VividPurple}{HTML}{238DFA}
\definecolor{SlateGrey}{HTML}{2E2E2E}
\definecolor{LightGrey}{HTML}{666666}
\colorlet{heading}{VividPurple}
\colorlet{accent}{VividPurple}
\colorlet{emphasis}{SlateGrey}
\colorlet{body}{LightGrey}

% Change the bullets for itemize and rating marker
% for \cvskill if you want to
\renewcommand{\itemmarker}{{\small\textbullet}}
\renewcommand{\ratingmarker}{\faCircle}

%% sample.bib contains your publications
\addbibresource{sample.bib}

\begin{document}

\name{Luyanda Mncube}
\photo{3cm}{me}
\tagline{Software Engineering Student}

% Cropped to square from https://en.wikipedia.org/wiki/Marissa_Mayer#/media/File:Marissa_Mayer_May_2014_(cropped).jpg, CC-BY 2.0
\personalinfo{%
  % Not all of these are required!
  % You can add your own with \printinfo{symbol}{detail}
  \email{lmncube@student.wethinkcode.co.za}
%   \phone{000-00-0000}
  \mailaddress{23 Siltstone Complex, Northgate 2162}
  \homepage{https://github.com/luyandamncube/}
  \linkedin{linkedin.com/in/luyanda-mncube-96bb54a0/}
%   \github{} % I'm just making this up though.
%   \orcid{orcid.org/0000-0000-0000-0000} % Obviously making this up too. If you want to use this field (and also other academicons symbols), add "academicons" option to \documentclass{altacv}
}

%% Make the header extend all the way to the right, if you want. Extend the right margin by 8cm (=6.8cm marginparwidth + 1.2cm marginparsep)
\begin{adjustwidth}{6cm}{-8cm}
\makecvheader
\end{adjustwidth}

%% Provide the file name containing the sidebar contents as an optional parameter to \cvsection.
%% You can always just use \marginpar{...} if you do
%% not need to align the top of the contents to any
%% \cvsection title in the "main" bar.

\cvsection[page1sidebar]{About me}
Inquisitive, hard-working and a teacher at heart. I love to learn and always find myself sharing everything I learn about technology with people around me. People come first in my life and I am not afraid of a challenge or shouldering responsibility. Previous to studying at wethinkcode, I worked 40 hours a week as a TEFL English teacher while studying computer science at UNISA. I have a strong foundation in data structures, algorithms, and problem solving.

\cvsection{Experience}

\cvevent{Online English Teacher}{EF Education First}{April 2016 - December 2017}{Bryanston, SA}
\begin{itemize}
\item I worked as an Online English Teacher to teach English as a Foreign language to students across the globe
\item We Utilized an online platform to present an prepare lessons to students from the EF Education First offices in
Bryanston, South Africa
\item Taught both private and group lessons
\item Prepared \& Evaluated lessons in \& TOEIC/TOEFL speaking  tests, English Fluency Testing and English Proficiency Testing

\end{itemize}

\divider

\cvevent{Team Elect}{Wethinkcode}{May 2018 -- ongoing}{Johannesburg, SA}
\begin{itemize}
\item Oversaw weekly team stand up meetings on campus with a group of 10 students
\item Attended fortnightly meetings with Campus manager \& House Guardians to address issues regarding student performance and welfare
\end{itemize}

\divider

\cvevent{National Senior Certificate}{Northcliff High School}{Class of 2011}{Johannesburg, SA}
\begin{itemize}
\item Academic Half Colours
\item 4 Distinctions
\item Representative Council of Learners
\end{itemize}

% \divider

% \cvevent{Product Engineer}{Google}{23 June 1999 -- 2001}{Palo Alto, CA}

% \begin{itemize}
% \item Joined the company as employe \#20 and female employee \#1
% \item Developed targeted advertisement in order to use user's search queries and show them related ads
% \end{itemize}

\cvsection{A Day of My Life}

% Adapted from @Jake's answer from http://tex.stackexchange.com/a/82729/226
% \wheelchart{outer radius}{inner radius}{
% comma-separated list of value/text width/color/detail}
\wheelchart{1.5cm}{0.5cm}{%
  10/13em/accent!30/Sleeping \& dreaming about solutions, 
  25/9em/accent!60/Working with other students to solve a problem,
  5/12em/accent!10/Reading news articles, 
  20/12em/accent!40/Research for projects,
  5/8em/accent!20/Self Study Session (Never too much studying!),
  30/9em/accent/Working on Coding Projects,
  5/8em/accent!20/Foodie time :)
}

\clearpage

\cvsection[page2sidebar]{Projects Completed}
\section*{libft}
\homepage{https://github.com/luyandamncube/libft}\\
- recreation of the C standard library\\
- used for all future wethinkcode projects in C
\section*{get\_next\_line}
\homepage{https://github.com/luyandamncube/get\_next\_line}\\
- recreation of get\_line function \\
- read from a specified file descriptor (stdin, stdout or stderr)\\
\section*{filler}
\homepage{https://github.com/luyandamncube/filler}\\
- game AI\\
- uses heuristic to battle other AI
\section*{push\_swap}
\homepage{https://github.com/luyandamncube/push\_swap}\\
- stack sorting (using two stacks)\\
- uses predefined instruction set for optimal solve
\section*{Lem-in}
\homepage{https://github.com/luyandamncube/Lem-in}\\
- graph traversal\\
- provide ants in an ant farm the most optimal route 

\cvsection{Presentations \& Articles}
\section*{\emph{The hitchhiker's guide to C}}
\homepage{https://github.com/luyandamncube/-42\_A\_hitchhikers\_guide\_to\_C}\\
\section*{\emph{Scrum - An introduction to Scrum Agile Methodology}}
\homepage{https://docs.google.com/presentation/d/1YcuTgwdasABdqbMF\_Ege13K-YWg3U2OCRb7XeiraNBs/edit?usp=sharing}\\


\end{document}
